%%%%%%%%%%%%%%%%%%%%%%%%%%%%%%%%%%%%%%%%%
% Medium Length Graduate Curriculum Vitae
% LaTeX Template
% Version 1.1 (9/12/12)
%
% This template has been downloaded from:
% http://www.LaTeXTemplates.com
%
% Original author:
% Rensselaer Polytechnic Institute (http://www.rpi.edu/dept/arc/training/latex/resumes/)
%
% Important note:
% This template requires the res.cls file to be in the same directory as the
% .tex file. The res.cls file provides the resume style used for structuring the
% document.
%
%%%%%%%%%%%%%%%%%%%%%%%%%%%%%%%%%%%%%%%%%

%----------------------------------------------------------------------------------------
%	PACKAGES AND OTHER DOCUMENT CONFIGURATIONS
%----------------------------------------------------------------------------------------

\documentclass[margin, 11pt]{res} % Use the res.cls style, the font size can be changed to 11pt or 12pt here

\usepackage[top=0.4in, bottom=0.5in, left=0.4in, right=0.4in]{geometry}
\usepackage{setspace}
\usepackage[adobe-utopia]{mathdesign}
\usepackage[T1]{fontenc}

\singlespacing
%\onehalfspacing
%\doublespacing
%\setstretch{1.1}
%\usepackage{helvet} % Default font is the helvetica postscript font
%\usepackage{newcent} % To change the default font to the new century schoolbook postscript font uncomment this line and comment the one above
\usepackage[colorlinks=True]{hyperref}
\usepackage{comment}
\setlength{\textwidth}{6.2in} % Text width of the document

\begin{document}

%----------------------------------------------------------------------------------------
%	NAME AND ADDRESS SECTION
%----------------------------------------------------------------------------------------

\centerline{\huge\bf Ajil Jalal} % Your name at the top
915 E 41 ST, APT 203 \hfill Email: ajiljalal@utexas.edu\\
Austin, TX- 78751\hfill Website: \href{https://sites.google.com/site/ajiljalal}{https://sites.google.com/site/ajiljalal}\\
\vbox{\hrule width\textwidth height 3pt}\smallskip % Horizontal line after name; adjust line thickness by changing the '1pt'
%\centerline{915 E 41 ST, APT 203} % Your address
%\centerline{Austin, Texas 78751}
%\centerline{Phone: (512) 888-7861 }
%\centerline{Email: ajiljalal@utexas.edu }
%\centerline{Website: \href{https://sites.google.com/site/ajiljalal}{https://sites.google.com/site/ajiljalal}}

%----------------------------------------------------------------------------------------
\vspace*{-13pt}
\begin{resume}

%----------------------------------------------------------------------------------------
%	OBJECTIVE SECTION
%----------------------------------------------------------------------------------------
 
\section{\large Education}
{\bf University of Texas at Austin} \hfill 2016-Present\\
{\sl M.S. and Ph.D., Electrical and Computer Engineering} \hfill GPA: 3.9/4.0 \\
{\sl Advisor}: Prof. Alexandros G. Dimakis \\
{\sl Interests}: Generative Models, Statistical Machine Learning, Information and Coding Theory

{\bf Indian Institute of Technology Madras} \hfill 2012-2016\\
{\sl Bachelor of Technology (Honours) in Electrical Engineering} \hfill GPA: 9.06/10 \\
{\sl Advisors}: Prof. Krishna Jagannathan and Prof. Rahul Vaze \\
{\sl Minor}: Systems Engineering 

\section{\large Publications} 
Ashish Bora, {\bf Ajil Jalal}, Eric Price, Alexandros G. Dimakis. ``Compressed Sensing Using Generative Models", \href{http://proceedings.mlr.press/v70/bora17a/bora17a.pdf}{ICML 2017} , Sydney, Australia.

Umang Bhaskar, {\bf Ajil Jalal}, Rahul Vaze. ``The Adwords Problem with Strict Capacity Constraints", \href{http://drops.dagstuhl.de/opus/volltexte/2016/6907/pdf/lipics-vol65-fsttcs2016-complete.pdf#page=365}{FSTTCS 2016} , Chennai, India.

\section{\large Preprints}
D. Van Veen, {\bf A. Jalal}, E. Price, S. Vishwanathan, and A.G. Dimakis. "Compressed Sensing Using Deep Image Prior and Learned Regularization." \href{https://arxiv.org/abs/1806.06438}{1806.06438} (2018).

A. Ilyas, {\bf A. Jalal}, E. Asteri, C. Daskalakis, and A.G. Dimakis. "The Robust Manifold Defense: Adversarial Training using Generative Models." \href{https://arxiv.org/abs/1712.09196}{arXiv:1712.09196} (2017). 

\section{\large Professional \\ Experience} 
{\bf Tata Institute of Fundamental Research} \hfill Mumbai, India\\
Undergraduate Research Intern\hfill Summer 2015\\
Designed approximation algorithms and showed approximation bounds for an online combinatorial optimization problem.

{\bf Audience Communication Systems} \hfill Bangalore, India\\
Undergraduate Intern \hfill Summer 2014\\
Worked on a text dependent automatic speaker recognition system.

{\bf Audience Communication Systems} \hfill Bangalore, India\\
Undergraduate Intern \hfill Winter 2013\\
Worked on reducing power dissipation in MIPS processors by minimising switching activity
in the processor.

\section{\large Projects}
{\bf The Robust Manifold Defense: Adv. Training Using Gen. Models}\hfill May 2017- Present\\
{\sl UT Austin, with Andrew Ilyas, Eirini Asteri, Prof. A.G. Dimakis, and Prof. C. Daskalakis}
\begin{itemize}\itemsep -2pt
	\item By adding imperceptible noise to a clean image, an adversary can arbitrarily influence the prediction of a neural network on the image. We show that generative models can defend against adversarial attacks.
	\item We search for an image in the span of a generative model that is close to an input image- this helps filter out adversarial perturbations. We also demonstrate how this idea can be used to robustify a classifier during its training.
\end{itemize}

{\bf Compressed Sensing Using Generative Models}\hfill August 2016- Present\\
{\sl UT Austin, with Ashish Bora, Prof. Alexandros G. Dimakis, and Prof. Eric Price}
\begin{itemize}\itemsep -2pt
	\item Introduced a new approach to compressed sensing. Traditional compressed sensing tries to find a sparse solution to an under-determined system of linear equations.
	\item Our approach is to search for an approximate solution in the span of a generative model.
	\item Proved upper bounds on number of measurements required for recovering a solution with low $\ell_2$ error. Empirical results show that we require $10$x less measurements than the traditional LASSO algorithm.
\end{itemize}

{\bf The Adwords Problem with Strict Capacity Constraints} \hfill May 2015- May 2016\\
{\sl TIFR, with Prof. Rahul Vaze and Prof. Umang Bhaskar}
\begin{itemize}\itemsep -2pt
	\item An adversary produces weighted jobs to a set of servers with finite capacities at discrete	time steps, and a matching must be found at each time step. Objective is to maximize the aggregate sum of jobs matched.
	\item Designed and proved approximation guarantees for randomised and deterministic online algorithms. Also showed that a load balancing algorithm is near-optimal for a special case.
	\item Proved lower bounds which show our algorithms are almost tight.
\end{itemize}

%\begin{comment}
{\bf Text Dependent Automatic Speaker Recognition} \hfill Summer 2014\\
{\sl Audience Communication Systems, with Murali Deshpande and Vinay N Krishnan}
\begin{itemize}\itemsep -2pt
\item Implemented an adaptive Gaussian Mixture Model which can be trained to recognise a particular
keyphrase by a 	user. Can be used as part of a voice activated wake up feature for
cellphones.
\item Model uses approximately 10 seconds of training data per user and achieves 80\%+ accuracy.
\end{itemize}
%\end{comment}

%----------------------------------------------------------------------------------------
%	PROFESSIONAL EXPERIENCE SECTION
%----------------------------------------------------------------------------------------

\section{\large Honors}
\begin{itemize}\itemsep -2pt
	\item Ranked {\bf 535} nationally in the {\bf 2012 IITJEE}, among 700,000 competitors.
	\item {\bf Karnataka Regional Mathematical Olympiad} scholar. Attended the {\bf Indian National Mathematical Olympiad (INMO)} camp and represented Karnataka
	in the INMO, 2011.
	\item {\bf Kishore Vaigyanik Protsahan Yojana (KVPY)} fellow, 2012.
	\item Honorable Mention in {\bf Quantify}, an analytics competition organised by Goldman-Sachs,
	2015.
	\item Nominated for the {\bf INSPIRE} scholarship, awarded to the top 1\% in the CBSE grade XII
	examinations, 2012.
	\item Ranked {\bf 63} in the {\bf Kerala Common Entrance Examination(CEE)}, 2012.
\end{itemize}

\section{\large Teaching \\ Experience}
{\bf University of Texas at Austin:}\\
{\sl Teaching Assistant}, EE351K: Introduction to Probability and Statistics \hfill Spring 2017\\
{\sl Teaching Assistant}, EE360C: Algorithms \hfill Fall 2016

\section{\large Skills}
{\sl Programming languages:} Python, C, C++.\\
{\sl Libraries and Toolkits:} Tensorflow, PyTorch, Matlab, \LaTeX, Numpy, Scipy.

\section{\large Relevant \\ Courses}

\setlength\tabcolsep{15pt}
\begin{tabular}{ll}
	Machine Learning & Information Theory \\
	Error Control Coding & Convex Optimization Theory and Algorithms\\
	Probability and Stochastic Processes & Approximation Algorithms \\ 
	Randomized Algorithms & Adaptive Signal Processing \\
	Pseudorandomness & Theory of Computation \\
	Digital Communication Systems & Computational Methods in EE \\ 
	Analog and Digital Signal Processing & Modern Control Theory\\
	Network Analysis & Multivariate Data Analysis\\
	Real Analysis & Complex Analysis\\
	Process Optimization & Reinforcement Learning\\
\end{tabular}

%\section{Hobbies}
%\textit{{\sl Music:} Professionally trained in Tabla and completed Preliminary, First Year and Second Year Examinations. All examinations were conducted by the {\bf Bangiya Sangeet Parishad}.\\
%{\sl Other:} Soccer}, squash, pencil sketching, watercolor, oil painting.



\end{resume}
\end{document}